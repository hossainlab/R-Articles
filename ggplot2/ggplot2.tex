\PassOptionsToPackage{unicode=true}{hyperref} % options for packages loaded elsewhere
\PassOptionsToPackage{hyphens}{url}
%
\documentclass[ignorenonframetext,aspectratio=169]{beamer}
\usepackage{pgfpages}
\setbeamertemplate{caption}[numbered]
\setbeamertemplate{caption label separator}{: }
\setbeamercolor{caption name}{fg=normal text.fg}
\beamertemplatenavigationsymbolsempty
% Prevent slide breaks in the middle of a paragraph:
\widowpenalties 1 10000
\raggedbottom
\setbeamertemplate{part page}{
\centering
\begin{beamercolorbox}[sep=16pt,center]{part title}
  \usebeamerfont{part title}\insertpart\par
\end{beamercolorbox}
}
\setbeamertemplate{section page}{
\centering
\begin{beamercolorbox}[sep=12pt,center]{part title}
  \usebeamerfont{section title}\insertsection\par
\end{beamercolorbox}
}
\setbeamertemplate{subsection page}{
\centering
\begin{beamercolorbox}[sep=8pt,center]{part title}
  \usebeamerfont{subsection title}\insertsubsection\par
\end{beamercolorbox}
}
\AtBeginPart{
  \frame{\partpage}
}
\AtBeginSection{
  \ifbibliography
  \else
    \frame{\sectionpage}
  \fi
}
\AtBeginSubsection{
  \frame{\subsectionpage}
}
\usepackage{lmodern}
\usepackage{amssymb,amsmath}
\usepackage{ifxetex,ifluatex}
\usepackage{fixltx2e} % provides \textsubscript
\ifnum 0\ifxetex 1\fi\ifluatex 1\fi=0 % if pdftex
  \usepackage[T1]{fontenc}
  \usepackage[utf8]{inputenc}
  \usepackage{textcomp} % provides euro and other symbols
\else % if luatex or xelatex
  \usepackage{unicode-math}
  \defaultfontfeatures{Ligatures=TeX,Scale=MatchLowercase}
\fi
\usefonttheme{structurebold}
% use upquote if available, for straight quotes in verbatim environments
\IfFileExists{upquote.sty}{\usepackage{upquote}}{}
% use microtype if available
\IfFileExists{microtype.sty}{%
\usepackage[]{microtype}
\UseMicrotypeSet[protrusion]{basicmath} % disable protrusion for tt fonts
}{}
\IfFileExists{parskip.sty}{%
\usepackage{parskip}
}{% else
\setlength{\parindent}{0pt}
\setlength{\parskip}{6pt plus 2pt minus 1pt}
}
\usepackage{hyperref}
\hypersetup{
            pdftitle={Introduction to Scientific Computing for Biologists},
            pdfborder={0 0 0},
            breaklinks=true}
\urlstyle{same}  % don't use monospace font for urls
\newif\ifbibliography
\setlength{\emergencystretch}{3em}  % prevent overfull lines
\providecommand{\tightlist}{%
  \setlength{\itemsep}{0pt}\setlength{\parskip}{0pt}}
\setcounter{secnumdepth}{0}

% set default figure placement to htbp
\makeatletter
\def\fps@figure{htbp}
\makeatother


\title{Introduction to Scientific Computing for Biologists}
\providecommand{\subtitle}[1]{}
\subtitle{ISCB20.09 - Introduction to Data Visualization with R\\
An Introduction to ggplot2}
\author{Md. Jubayer Hossain\\
\url{https://jhossain.me/}~\\
\href{mailto:jubayer@hdrobd.org}{\nolinkurl{jubayer@hdrobd.org}}}
\providecommand{\institute}[1]{}
\institute{Founder\\
Health Data Research Organization\\
Lead Organizer\\
Scientific Computing for Biologists}
\date{13 March 2021}

\begin{document}
\frame{\titlepage}

\begin{frame}[t]{What is ggplot2?}
\protect\hypertarget{what-is-ggplot2}{}

\begin{itemize}
\tightlist
\item
  ggplot2 is a data visualization package for the statistical
  programming language R.
\item
  Created by Hadley Wickham in 2005, ggplot2 is an implementation of
  Leland Wilkinson's Grammar of Graphics---a general scheme for data
  visualization which breaks up graphs into semantic components such as
  scales and layers.
\end{itemize}

\end{frame}

\begin{frame}[t]{Types of Visualization}
\protect\hypertarget{types-of-visualization}{}

In statistics, we generally have two kinds of visualization:

\begin{itemize}
\tightlist
\item
  Exploratory data visualization: Exploring the data visually to find
  patterns among the data entities.
\item
  Explanatory data visualization: Showcasing the identified patterns
  using simple graphs.
\end{itemize}

\end{frame}

\begin{frame}[t]{Why Visualization?}
\protect\hypertarget{why-visualization}{}

``A picture is worth a thousand words''

\begin{itemize}
\tightlist
\item
  Data visualizations make big and small data easier for the human brain
  to understand, and visualization also makes it easier to detect
  patterns, trends, and outliers in groups of data.
\item
  Good data visualizations should place meaning into complicated
  datasets so that their message is clear and concise.
\end{itemize}

\end{frame}

\begin{frame}[t]{Grammar of Graphics}
\protect\hypertarget{grammar-of-graphics}{}

\begin{itemize}
\tightlist
\item
  Data - The dataset being plotted
\item
  Aesthetics- The scales onto which we plot our data
\item
  Geometry - The visual elements used for our data
\item
  Facet -Groups by which we divide the data
\end{itemize}

\end{frame}

\begin{frame}[t]{Data Import}
\protect\hypertarget{data-import}{}

\begin{itemize}
\tightlist
\item
  readr - \url{https://readr.tidyverse.org/}
\item
  readxl- \url{https://readxl.tidyverse.org/}
\item
  haven - \url{https://haven.tidyverse.org/}
\end{itemize}

\end{frame}

\begin{frame}[t]{Data Manipulation}
\protect\hypertarget{data-manipulation}{}

\begin{itemize}
\tightlist
\item
  tidyr- \url{https://tidyr.tidyverse.org/}
\item
  dplyr- \url{https://dplyr.tidyverse.org/}
\end{itemize}

\end{frame}

\begin{frame}{পাইথন এবং আর}
\protect\hypertarget{ux9aaux9beux987ux9a5ux9a8-ux98fux9acux982-ux986ux9b0}{}

\end{frame}

\begin{frame}[t]{Data Sources}
\protect\hypertarget{data-sources}{}

\begin{itemize}
\tightlist
\item
  \url{https://genome.ucsc.edu/}
\item
  \url{https://www.ensembl.org/index.html}
\item
  \url{https://www.encodeproject.org/}
\end{itemize}

\end{frame}

\end{document}
